\documentclass[14pt]{extarticle}
\usepackage[paper=a4paper, top=2cm, bottom=2cm, right=1cm, left=3cm]{geometry}
\usepackage[utf8]{inputenc}
\usepackage[british]{babel}

\linespread{1.3}
\setlength{\parindent}{0pt}
\setlength{\parskip}{1em}

\begin{document}
\section{Annotation}
Kairanbay M. \textit{A  Review and  Evaluations of Shortest  Path Algorithms}, International Journal of Scientific and  Technology Research ,Volume 2, Issue 6 June 2013. 
%https://www.researchgate.net/publication/310594546_A_Review_and_Evaluations_of_Shortest_Path_Algorithms

Algorithms that solve the shortest path tools are used in many industries, in particular when setting up computer networks.
They help reduce routing time.
The main purpose of the article is to evaluate a number of algorithms that solve the shortest path problem.
In particular, the following algorithms are considered: the Dijkstra algorithm, the Floyd-Warshall algorithm, the Bellman-Ford algorithm, and the genetic algorithm.
Illustrations were given to describe the main idea of the algorithms.
The structure of optimal paths resulting from the application of a genetic algorithm is presented.
The main differences of the algorithms from each other and their evaluation are described.
The time complexity of the presented algorithms is calculated.

\section{Annotation}
Elmasry, A. \& Shokry, A. \textit{A new algorithm for the shortest-path problem.} Networks, 74, December 2018. 

The paper proposes a new algorithm for solving the shortest path problem.
The necessary theorems substantiating the convergence of the considered method are proved.
For the constructed algorithm, implementation on pseudocode is presented.
The main classical algorithms for solving the shortest path problem and algorithms for finding negative cycles are described.
The proposed algorithm is an improvement of Dijkstra's algorithm.
The practical results of the program work are presented.
The time complexity of the developed algorithm is compared with the well-known classical algorithms for solving the shortest path problem.
The theoretical complexity of the algorithm coincides with the complexity of the Bellman--Ford--Moore algorithm.
According to the results of the practical part, it turned out that in practice the proposed method works faster.

\section{Annotation}
Ioachim, I. \& Gélinas, S. \& Soumis, F. \& Desrosiers, J. \textit{A Dynamic Programming Algorithm for the Shortest Path Problem with Time Windows and Linear Node Costs.} Networks, 31, 193-204, 1998.

The article presents a way to apply the dynamic programming method to solve the optimal control problem with time windows and linear costs for node processing.
The dynamic programming method was proposed by R. Bellman and is used to find optimal paths in various applications.
Based on the Bellman algorithm, an algorithm is proposed that takes into account the cost of linear nodes.
This solution is applicable in the planning of air flights and the distribution of work shifts.
The algorithmic complexity of the algorithm was calculated.
To prove the effectiveness of the method, a comparison was made with a similar algorithm based on partial discretization of time windows.
As a result, we found that the method outperforms existing algorithms in the case of wide time windows, as well as with a large number of connections with negative weights.

\section{Annotation}
Ferone, D. \& Festa, P. \& Fugaro, S. \& Pastore, T. \textit{A dynamic programming algorithm for solving the k-Color Shortest Path Problem.} Optimization Letters, 15, 1-20, 2021.

Many algorithms for solving the shortest path problem are described in the literature.
This article deals with the problem of the shortest path in a colored graph.
This problem arises when designing large telecommunication networks.
This statement makes it possible to guarantee the optimal transition between the nodes of the graph along the allowed edges.
This takes into account the requirement for the reliability of the constructed network architecture while optimizing time and money costs.
This paper describes an algorithm based on the dynamic programming method.
The proposed algorithm is compared with the methods of vertices and boundaries constructed earlier by the author.
Numerical results of the algorithm operation are presented, according to which the dynamic programming method is superior in performance to the previously studied algorithms.

\section{Annotation}
Zhao, Z. \& Zhu, Q. \& Guo, B. \textit{Robust Trajectory Tracking of Uncertain Systems via Adaptive Critic Learning.} Complexity, 2022.

In this article, a controller (optimal control in the form of a strategy) is implemented for the problem of robust trajectory tracking with uncertainty.
To solve the problem, an extended system is introduced, which includes a deviation from the desired trajectory.
The resulting extended problem is solved by the dynamic programming method.
The representation of a problem with uncertainty in the form of an extended deterministic problem is a scientific novelty.
The method of solving the Hamilton--Jacobi--Bellman equation, which is involved in the dynamic programming method, was chosen to train the neural network.
Two innovative methods for training a neural network are proposed.
The convergence of learning as a selection of unknown weights of a neural network is proved.
The algorithm was implemented, and the results of its work are presented, showing the time efficiency of the proposed solution method.
\end{document}